% Created 2016-09-21 Wed 08:46
\documentclass[11pt]{article}
\usepackage[utf8]{inputenc}
\usepackage[T1]{fontenc}
\usepackage{fixltx2e}
\usepackage{graphicx}
\usepackage{grffile}
\usepackage{longtable}
\usepackage{wrapfig}
\usepackage{rotating}
\usepackage[normalem]{ulem}
\usepackage{amsmath}
\usepackage{textcomp}
\usepackage{amssymb}
\usepackage{capt-of}
\usepackage{hyperref}
\author{Jason Lewis}
\date{\today}
\title{Lazy Programmers Guide to html tables in Dancer2}
\hypersetup{
 pdfauthor={Jason Lewis},
 pdftitle={Lazy Programmers Guide to html tables in Dancer2},
 pdfkeywords={},
 pdfsubject={},
 pdfcreator={Emacs 25.1.50.1 (Org mode 8.3.6)}, 
 pdflang={English}}
\begin{document}

\maketitle
\setcounter{tocdepth}{1}
\tableofcontents


\section*{About me}
\label{sec:orgheadline5}

Jason Lewis

Owner of a wholesale distribution business in Australia

Jack of all trades…
\subsection*{contact}
\label{sec:orgheadline1}
\begin{itemize}
\item email: jason@dickson.st
\item IRC: k-man
\item Twitter: @jasonblewis
\end{itemize}


\subsection*{Why am I doing this?}
\label{sec:orgheadline2}

Business reporting done in Crystal Reports

\begin{itemize}
\item Replace Crystal Reports with something less proprietary.
\item make reports accessable on the web
\end{itemize}

\subsection*{Crystal Reports}
\label{sec:orgheadline3}
Existing system was built in Crystal Reports
\begin{itemize}
\item Pros: 
\begin{itemize}
\item very quick to build reports
\item quite easy to customise and make the reports just so
\end{itemize}
\item Cons: 
\begin{itemize}
\item proprietary
\item distribution of reports difficult
\item not easy to put the reports on the web
\item win32 only
\end{itemize}
\end{itemize}
\subsection*{Why Dancer2?}
\label{sec:orgheadline4}

I looked around 

Dancer has an awesome community.

\section*{Hand coded html table in Template}
\label{sec:orgheadline10}
\subsection*{Route handler}
\label{sec:orgheadline6}
\begin{verbatim}
get '/demo01' => sub {

  my $sth = database->prepare(
	'select * from invoices',
      );
  $sth->execute() or die $sth->errstr;

  my $fields = $sth->{NAME};
  my $invoices = $sth->fetchall_arrayref({});

  template 'demo01',
    { title => 'demo01',
      fields => $fields,
      invoices => $invoices,
    };
};
\end{verbatim}
\subsection*{View}
\label{sec:orgheadline7}
\begin{verbatim}
<table width="80%" id="example-table">
...
    <tbody>
	[% FOREACH invoice IN invoices %]
	    <tr>
		[% FOREACH field IN fields %]
		    <td>[% invoice.$field %]</td>
		[% END %]
	    </tr>
	[% END  %]
    </tbody>
</table>
\end{verbatim}
\subsection*{Demo 1}
\label{sec:orgheadline8}
\url{http://localhost:5000/demo01}
\subsection*{Pros and Cons}
\label{sec:orgheadline9}
\begin{itemize}
\item pros:
\begin{itemize}
\item simple to write
\end{itemize}
\item Cons:
\begin{itemize}
\item resulting web page very static
\item tables with many rows cumbersome
\end{itemize}
\end{itemize}


\section*{add DataTables to the mix}
\label{sec:orgheadline18}
hand coded table + DataTables
\subsection*{What is DataTables?}
\label{sec:orgheadline11}
\begin{itemize}
\item JavaScript framework for manipulating html tables
\item adds about about 136k to your page (depending on components)
\item \url{https://datatables.net/}
\end{itemize}
\subsection*{Include the CSS and JavaScript for DataTables}
\label{sec:orgheadline12}
in the main.tt layout add:
\begin{verbatim}
<!-- DataTables -->
<link rel="stylesheet" 
      href="[% request.uri_base %]/css/jquery.dataTables.min.css">
<script type="text/javascript" 
	src="javascripts/jquery.dataTables.min.js">
</script>
\end{verbatim}

\subsection*{call the DataTable enabler after document ready}
\label{sec:orgheadline13}
add JavaScript DataTable call to the view:
\begin{verbatim}
<script type="text/javascript">
 $(document).ready(function(){
     $('#example-datatable').DataTable();
 });
</script>
\end{verbatim}
\#example-datatable is the CSS id of the table you want to make fancy
\subsection*{add \#example CSS id to table}
\label{sec:orgheadline14}
\begin{verbatim}
<table width="80%">
    <thead>
	<tr>
	    [% FOREACH field IN fields %]
		<th>[% field %]</th>
	    [% END  %]
	</tr>
    </thead>
    ...
</table>
\end{verbatim}
\subsection*{add \#example CSS id to table}
\label{sec:orgheadline15}
\begin{verbatim}
<table width="80%" id="example-datatable">
    <thead>
	<tr>
	    [% FOREACH field IN fields %]
		<th>[% field %]</th>
	    [% END  %]
	</tr>
    </thead>
    ...
</table>
\end{verbatim}
\subsection*{Demo 02}
\label{sec:orgheadline16}
\url{http://localhost:5000/demo02}

\subsection*{Pros and Cons of hand coded html tables with DataTables}
\label{sec:orgheadline17}
\begin{itemize}
\item pros:
\begin{itemize}
\item very configurabe, you can generate the HTML table just how you like it.
\item easy to give rows and columns custom css IDs and classes
\end{itemize}
\item cons:
\begin{itemize}
\item not very reusable, you have to hand code each each report
\item changes to data structure may require updates to the view
\end{itemize}
\end{itemize}
\section*{using JSON with DataTables}
\label{sec:orgheadline23}
\begin{itemize}
\item build the HTML table headings in javascript
\item pass in a URL that returns JSON to DataTables
\item DataTables retrieves the data and fills the table.
\end{itemize}

\subsection*{build the table header in JavaScript}
\label{sec:orgheadline19}
insert the \#tableDiv
\begin{verbatim}
$( document ).ready( function( $ ) {
    $.ajax({
	"url": '[% json_data_url %]',
	"success": function(json) {
	    var tableHeaders = '';  
	    $.each(json.columns, function(i, val){
		tableHeaders += "<th>" + val.data + "</th>";
	    });

	    $("#tableDiv").html(
	      '<table id="displayTable"      \
		class="display compact"      \
		cellspacing="0"><thead><tr>'
	      + tableHeaders + '</tr></thead></table>');
	    $('#displayTable').DataTable(json);
	},
	"dataType": "json"
    });
});
\end{verbatim}
\subsection*{build a json route}
\label{sec:orgheadline20}
Include the columns you want to render and the results from the query
\begin{verbatim}
get '/api/demo03' => sub {
# return query as JSON
  my $sth = database->prepare(
	'select * from invoices',
      );
  $sth->execute() or die $sth->errstr;

  my $invoices = $sth->fetchall_arrayref({});

  send_as JSON => { columns => [
    { data => 'InvoiceId'},
    { data => 'InvoiceDate'},
    { data => 'CustomerId' },
    { data => 'BillingAddress'}
      ],
    data => $invoices,
  };
};
\end{verbatim}
\subsection*{Demo 03}
\label{sec:orgheadline21}
\url{http://localhost:5000/demo03}
\subsection*{Pros and Cons}
\label{sec:orgheadline22}
\begin{itemize}
\item pros
\begin{itemize}
\item very easy to reuse code
\item page response feels faster for the user
\end{itemize}
\item cons
\begin{itemize}
\item you need an API route to return the data
\item more difficult to customise your resulting html table
\item adding custom CSS IDs to rows requires writing javascript
\end{itemize}
\end{itemize}
\section*{styling the table}
\label{sec:orgheadline28}
DataTables comes with some predefined CSS
for example, classes for left and right alignment:
\begin{itemize}
\item dt-left
\item dt-right
\end{itemize}
\subsection*{css classes}
\label{sec:orgheadline24}
Add CSS classes to columns
\begin{verbatim}
…
send_as JSON => { columns => [
  { className => 'dt-right', data => 'InvoiceId',      },
  { className => 'dt-left',  data => 'InvoiceDate',    },
  { className => 'dt-right', data => 'CustomerId',     },
  { className => 'dt-left',  data => 'BillingAddress',
       title => 'Billing Address'}
    ],
  data => $invoices,
};
\end{verbatim}
\subsection*{Demo 04}
\label{sec:orgheadline25}
\url{http://localhost:5000/demo04}
\subsection*{Other columns properties}
\label{sec:orgheadline26}
columns has many other properties that can be useful
\begin{itemize}
\item name: Descriptive name for the column
\item title: Column title
\item visible: enable or disable display of this column
\end{itemize}
\subsection*{problem with this approach}
\label{sec:orgheadline27}
formatting creeping into the data view
\section*{table export options}
\label{sec:orgheadline38}
\subsection*{Users are never satisfied}
\label{sec:orgheadline29}
\begin{itemize}
\item Can I export it to Excel?
\item DataTables makes that easy
\item Buttons component.
\end{itemize}

\subsection*{CSS and JavaScript for DataTables Buttons}
\label{sec:orgheadline30}
\subsection*{Install pdfmake}
\label{sec:orgheadline31}
\begin{verbatim}
cd MyApp/public
bower install pdfmake
\end{verbatim}

\subsection*{add the DataTables Buttons css}
\label{sec:orgheadline32}
\begin{verbatim}
<link rel="stylesheet" 
  type="text/css" 
  href="https://cdn.datatables.net/buttons/1.1.1/css/buttons.dataTables.min.css">
\end{verbatim}

\subsection*{add the JavaScript}
\label{sec:orgheadline33}
\begin{verbatim}
<script src="/javascripts/buttons.html5.min.js"></script>
<script src="/javascripts/buttons.print.min.js"></script>
<script src='/bower_components/pdfmake/build/pdfmake.min.js'></script>
<script src='/bower_components/pdfmake/build/vfs_fonts.js'></script>
\end{verbatim}
\subsection*{add buttons option to our javascript}
\label{sec:orgheadline34}
Add this to our JavaScript from before
\begin{verbatim}
json.dom = 'Blfrtip'; // customise the table
json.buttons = ['copy',
		'csv',
		'excel',
		{ extend: 'pdfHtml5',
		  text: 'PDF',
		  orientation: 'landscape',
		  pageSize: 'A4',
		  download: 'download',
		  filename: '*',
		  extension: 'pdf'
		},
		'print'];
\end{verbatim}
\subsection*{Demo 05}
\label{sec:orgheadline35}
\url{http://localhost:5000/demo05}

\subsection*{As yet unresolved challenges}
\label{sec:orgheadline36}
\begin{itemize}
\item move formatting and options out of main route
\item formatting dates like '2009-01-01 00:00:00'
\item rounding floats to fixed decimal places
\end{itemize}
\subsection*{thanks for listening}
\label{sec:orgheadline37}
\end{document}